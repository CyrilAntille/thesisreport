
\chapter{Conclusion}
\label{chap:conclusion}
Conventional 2D medical ultrasound imaging is based on transmitting focused acoustic beams and recording their echoes from potential scatterer points in the imaged medium.
Traditionally beams are transmitted sequentially towards a set of focus points at a fixed distance, or radius, to the array's center.
The transmission process can be iterated for different radii, also known as focus lines.
For each transmit beam, the array is often dynamically focused towards the same focus point as the transmit beam and the resulting image samples contain information about physical elements in the medium on, or close to, the transmit/receive beam trajectory.
Many different image acquisition processes have been presented over the years as alternatives to this process. Different approaches can for example transmit multiple beams in parallel, create multiple receive beams per transmit beam or combine multiple receive beams in order to extract spatial information outside of their focus trajectories.

In general, the backscatterered signals from scatterer points that are not on a transmit and/or receive beam trajectory can be attenuated and cause scalloping loss, although different beamformers have different sensitivities to that effect.
Scalloping loss can become problematic when its magnitude is high enough to cause visible dim of the scatterer points' gain and, in extreme cases, result in them disappearing in the background.
When producing series of images in presence of tissue motion, scalloping loss is known to potentially result in scatterer points visibly blinking.

If a beamformer system is built such that the magnitude of scalloping loss is guaranteed to be below a visibility threshold in all cases, the system is said to be lateral shift-invariant.
When the effects of scalloping loss are visible, the system is often said to be subject to angular undersampling, which means that the density of transmit and/or receive beams is too low.
We showed that all beamformers used in this thesis, including the IAA-MB approaches, can be made lateral shift-invariant with a sufficiently-high beam density.
Furthermore, the IAA-MB approaches, despite being high-resolution beamformers, require significantly lower beam densities than that of the MV beamformer.

Most of the existing studies on motion in ultrasound imaging focus on the effect of motion on multiple image frames.
A high transmit beam density is desirable to limit the scale of scalloping loss, but can result in a high image acquisition than and thus impact the frame rate capacity of a beamformer.
Another aspect to motion that is often overlooked is its effects within a single frame.
Since the acquisition of a single frame is not instantaneous, motion in the imaged medium can also result in artifacts due to position shifts between each beam transmission.
The pulse signal transmitted by the probe is typically very short (about $1~\mu$s in this thesis), so any scatterer point with realistic velocity ($\boldsymbol{v}_s << c$, where $c$ is the speed of propagation of the transmitted signal) can be considered idle while reflecting it.
The main artifact of motion within a single frame is possible distortion of the scatterer point's shape, mostly in the form of dilation or erosion, depending on how many transmit beams hitting it.
With a relatively high image acquisition time, such distortions can become visible and result in image resolution loss.
Singlebeam beamformers are relatively robust to that aspect of motion since each image cell is only based on a transmit single beam.

Multibeam beamformers produce images whose cells can be computed from multiple transmit and/or receive beams. Due to that property, we predicted multibeam beamformers to be more sensitive to tissue motion within a single frame than singlebeam beamformers.
We showed that the IAA-MB beamformers are indeed more prone to distortions of the scatterer point's shape than the singlebeam DAS and MV beamformers.
In extreme cases, the resolution capacity of the IAA-MB beamformers can drop to that of DAS or even worst for the IAA-MBMB approach.

Since in real applications the imaged medium properties are not under our control, the image acquisition time should be kept as low as possible in order to attenuate any distortion of the scatterer points' shapes.
It seems that there is a choice to make between reducing distortion effects, by keeping the transmit beam density low, and reducing the effects of scalloping loss, by ensuring a transmit beam density high enough. Both issues can also be solved with a low density of wide transmit beams, but to the cost of reduced image resolution.
An alternative option used in this thesis is multi-line acquisition (MLA), which is the concept of creating multiple receive beams for each transmit beam.
Since the transducer arrays used in medical ultrasound imaging use dynamic focusing on receive, an increase of the receive beam density mostly costs computational time.
In most systems, the data recording and processing can be done simultaneously. Furthermore, data processing can often be made faster than data recording, which is limited by the propagation speed of the transmitted pulses.

Assuming a transmit beam density high enough to allow for scalloping loss correction, a MLA beamformer can be made shift-invariant with a lower image acquisition time, and therefore higher robustness to tissue motion, than that of its SLA variant with same receive beam density.
Given the probe and medium parameters as defined in Section \ref{sec:sim_param}, we estimated the minimum transmit beam density required for the DAS beamformer using SLA to be considered lateral shift-invariant and took it as reference for the all beamformers.
We then assumed that it is possible to make the MV and IAA-MB beamformers lateral shift-invariant with the same transmit beam density as SLA DAS and a smart use of the MLA approach.
Several MLA approaches have be presented over time and some are more efficient than others to correct for scalloping loss for receive beams that are not focused to the same point than the transmit beam they are based on.
In Section \ref{sec:naive_mla}, we found out that even a naive approach to MLA, without correction for scalloping loss, can help to improve the maximum frame rate of a lateral shift-invariant beamformer.
In addition, the IAA-MBMB beamformer impressively proved to be able to automatically correct for scalloping loss and to give similar performances than with perfect MLA.
It is on the other hand more sensitive than the other beamformers to motion within frames, although it might be of little consequences for realistic scatterer point velocities, depending on the image acquisition time.

The IAA-MB beamformers are emerging as robust adaptive beamformers.
They also are qualified as parameter-free algorithms. With some smart pre-calibration, we think that they they could become as easy to use as DAS and produce high-resolution images with similar, if not better, frame rate and robustness to DAS.
With some more studies and testing on real images, the IAA-MB beamformers might come out as strong alternatives to DAS.

